\documentclass{article}

\usepackage{lmodern}
\usepackage[T1]{fontenc}
\usepackage[spanish,activeacute]{babel}
\usepackage{mathtools}
\usepackage{graphicx}
\usepackage{geometry,pdfpages}
\usepackage{wrapfig}
\usepackage{enumerate} 
\usepackage{parskip}


\geometry{
	letterpaper,%showframe,
	left=30mm,top=10mm,right=30mm,bottom=10mm,
	headheight=6mm,headsep=7mm,foot=5mm,footskip=15mm,
	includeheadfoot
}

\setlength{\parindent}{0cm}


\begin{document}

		
	\begin{wrapfigure}[0]{r}{0.15\textwidth}
		\vspace{-142pt}
		\includegraphics[width=2cm]{loguito}
	\end{wrapfigure}
	
	\begin{center}
		\fbox{\parbox[t]{\linewidth}{
				\begin{flushleft}
					Universidad Centroamericana "José Simeón Cañas" \linebreak
					Facultad de Ingeniería y Arquitectura\linebreak
					Departamento de Electrónica e Informática\linebreak
					\textbf{Materia: }An'alisis Num'erico\linebreak
					\textbf{Catedr'atico: } Daniel Sosa\linebreak
					\textbf{Instructor: } Kevin Lopez\linebreak
					\textbf{Estudiante: } Elsy Alejandra Chavez Mendoza ($\texttt{00125717}$)\linebreak
					\textbf{Estudiante: } Fredy Alexander Sanchez Perez ($\texttt{00082817}$)\linebreak
					\textbf{Estudiante: } Erick Fernando Leones Arevalo ($\texttt{00092217}$)\linebreak
					\textbf{Estudiante: } German Alexander Castro Portillo ($\texttt{00229017}$)\linebreak
					Viernes 28 de junio de 2019
				\end{flushleft}
		}}
	\end{center}
	
	\textbf{Resumen. } \emph{Aqu'i ir'a el resumen}
	
		\section*{Definiciones}
			\large{
				El m'etodo de cuadratura de Gauss es un m'etodo num'erico para evaluar integrales definidas de funciones, por medio de sumatorias f'aciles de implementar, adem'as de aplicar los polinomios ortogonales.
				
				La regla del trapecio (Newton- Cotes para $\mathrm{n = 1}$) tiene grado de precisión uno. La regla de Simpson ($\mathrm{n = 2}$) es correcta hasta los polinomios de tercer grado inclusive. A diferencia de Newton Cotes la cuadratura de Gauss posee un grado de precisi'on de $\mathrm{2n-1}$, adem'as que selecciona los puntos de evaluaci'on de manera 'optima y no de forma igualmente espaciada.
				
				Los nodos $\mathrm{x_{1}, x_{2},..., x_{n}}$ en el intervalo [a,b] y los coeficientes c1, c2,..., cn, son elegidos para minimizar el error obtenido en la aproximaci'on.
				
				F'ormula de la Cuadratura Gaussiana: 
				\begin{equation}\label{eq:cuadGauss}
					\int_{-1}^{1} f(x) \cdot dx \approx \sum_{i=1}^{n} c_{i} f(x_{i})
				\end{equation}
				}

				\subsection*{Polinomios de Legendre}
				Existe un método por el cual se puede determinar los nodos y coeficientes  para las f'ormulas que dan el resultado exacto de los polinomios de grado superior. Se consideran varias colecciones de polinomios ortogonales, los cuales son funciones que tienen la propiedad de que una integral definida del producto de dos de ellos es igual a cero. Esta colecci'on de polinomios son llamados polinomios de Legendre, que van desde $\mathrm{P_{0}}$ hasta $\mathrm{P_{n}}$ y tienen las propiedades:
				\begin{enumerate}[1.]
					\item 
					Para cada $\mathrm{n}$, $\mathrm{P_{n}(x)}$ es un polinomio m'onico (polinomio de una variable cuyo coeficiente principal es 1) con grado $n$.
					\item
					$\mathrm{\int_{-1}^{1}P(x)P_{n}(x)\cdot dx = 0}$ cuando $\mathrm{P(x)}$ sea un polinomio de grado menor a $\mathrm{n}$
				\end{enumerate}
			
				Los primeros polinomios de Legendre son:\\
				$\mathrm{P_{0}(x)=1}$\\
				$\mathrm{P_{1}(x)=1}$\\
				$\mathrm{P_{2}(x)=x^{2}-\frac{3}{5}x}$\\
				$\mathrm{P_{3}(x)=x^{3}-\frac{6}{7}x^{2}+\frac{3}{35}}$\\
				De los polinomios anteriores se consiguen las siguientes ra'ices y coeficientes:\\
				
				\begin{center}
					\includegraphics[width=15cm]{roots}
					\par
					\vspace{0.2cm}
				\end{center}
				
	
		\section*{Resultados te'oricos}
		\section*{Resultados pr'acticos}
		
		\begin{thebibliography}{00}
			\bibitem{1} tengo sue'no xd
		\end{thebibliography}
	
\end{document}